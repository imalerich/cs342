\documentclass{jhwhw}
\author{Ian Malerich}
\title{Com S 342: Lambda Calculus \& Racket}
\usepackage{amssymb}

\begin{document}

\problem{}

Given

\begin{tabular}{r@{:}l}
	0&\ $\lambda$f.$\lambda$x.x \\
	succ&\ $\lambda$n.$\lambda$f.$\lambda$x.(f ((n f) x)) \\
	n&\ if n is a natural number then its semantics is the result of n applications of succ on 0. \\
	true&\ $\lambda$x.$\lambda$y.x \\
	false&\ $\lambda$x.$\lambda$y.y \\
	second&\ $\lambda$x.$\lambda$y.$\lambda$z.y \\
	g&\ $\lambda$n.((n second) false) \\
\end{tabular}
\bigbreak
\raggedright
What is the result of

\begin{enumerate}
	\item (g n) when n is 0.
	\item (g n) when n results from some application succ on 0.
	\item What mathematical/logical operation is computed by g.
\end{enumerate}

\solution

% Part A
\part

\raggedright
	(g 0) \\
	(g $\lambda$f.$\lambda$x.x) \\ 
	(\lambda $n.((n second) false) $ $\lambda$f.$\lambda$x.x) \\
	(($\lambda$f.$\lambda$x.x second) false) \\
	($\lambda$x.x false) \\
	false \\

% Part B
\part

	First consider what is happening when n applications of succ are taken on 0. \\
	\bigbreak
	Consider n = 1 application of succ on 0. \\
	(succ 0) \\
	($\lambda$n.$\lambda$f.$\lambda$x.(f ((n f) x)) $\lambda$f.$\lambda$x.x) \\
	$\lambda$f.$\lambda$x.(f (($\lambda$f.$\lambda$x.x f) x) \\
	$\lambda$f.$\lambda$x.(f ($\lambda$x.x x)) \\
	$\lambda$f.$\lambda$x.(f x) \\

	\bigbreak
	Now consider n = 2 applications of succ on 0. \\
	(succ (succ 0)) \\
	(succ $\lambda$f.$\lambda$x.(f x)) \\
	($\lambda$n.$\lambda$f.$\lambda$x.(f ((n f) x)) $\lambda$f.$\lambda$x.(f x)) \\
	$\lambda$f.$\lambda$x.(f (($\lambda$f.$\lambda$x.(f x) f) x)) \\
	$\lambda$f.$\lambda$x.(f ($\lambda$x.(f x) x)) \\
	$\lambda$f.$\lambda$x.(f (f x)) \\

	\bigbreak
	As discussed in class, this pattern will continue. \\
	That is, n applications of succ on 0 can be written as the following. \\
	$\lambda$f.$\lambda$x.($f^1$ ($\ldots$ ($f^n$ x))) \\
	Note the superscript denotes the occurrence index for the lambda function f. \\

	\bigbreak
	Now consider (g n) \\
	(g n) \\
	(g (succ_1$ (\ldots (succ_n$ 0))))$ \\
	(\lambda $n.((n second) false) (succ_1$ (\ldots (succ_n$ 0))))$ \\
	(((succ_1$ (\ldots (succ_n$ 0))) second) false)$ \\
	(($\lambda$f.$\lambda$x.($f^1$ ($\ldots$ ($f^n$ x))) second) false)\\
	(($\lambda$f.$\lambda$x.($f^1$ ($\ldots$ ($f^n$ x))) $\lambda$x.$\lambda$y.$\lambda$z.y) false)\\
	As these are lambda expressions, subscripts denote both the number of occurrences as well as 
	the uniqueness of each lambda variable. \\
	($\lambda$x.($\lambda$x$_1$.$\lambda$y$_1$.$\lambda$z$_1$.y$_1$ 
		($\ldots$ ($\lambda$x$_n$.$\lambda$y$_n$.$\lambda$z$_n$.y$_n$ x))) false) \\
	($\lambda$x$_1$.$\lambda$y$_1$.$\lambda$z$_1$.y$_1$ 
		($\ldots$ ($\lambda$x$_n$.$\lambda$y$_n$.$\lambda$z$_n$.y$_n$ false))) \\
	($\lambda$x$_1$.$\lambda$y$_1$.$\lambda$z$_1$.y$_1$ ($\ldots$ 
		($\lambda$x$_{n-1}$.$\lambda$y$_{n-1}$.$\lambda$z$_n$.y$_{n-1}$ 
		($\lambda$x$_n$.$\lambda$y$_n$.$\lambda$z$_n$.y$_n$ 
		false))) \\
	($\lambda$x$_1$.$\lambda$y$_1$.$\lambda$z$_1$.y$_1$ ($\ldots$ 
		($\lambda$x$_{n-1}$.$\lambda$y$_{n-1}$.$\lambda$z$_n$.y$_{n-1}$ 
		$\lambda$y$_n$.$\lambda$z$_n$.y$_n$))) \\
	Note that $\lambda$y.$\lambda$z.y is defined as `true'. \\
	($\lambda$x$_1$.$\lambda$y$_1$.$\lambda$z$_1$.y$_1$ ($\ldots$ 
		($\lambda$x$_{n-1}$.$\lambda$y$_{n-1}$.$\lambda$z$_n$.y$_{n-1}$ 
		true))) \\
	If n = 1 we would be done here and have a result of 'true' \\ (as the 1$\ldots$n-1 terms would not exist). \\
	If n $\textgreater$ 1, we apply another beta reduction \\
	($\lambda$x$_1$.$\lambda$y$_1$.$\lambda$z$_1$.y$_1$ ($\ldots$ 
		($\lambda$x$_{n-2}$.$\lambda$y$_{n-2}$.$\lambda$z$_n$.y$_{n-2}$ 
		$\lambda$y$_{n-1}$.$\lambda$z$_n$.y$_{n-1}$)) \\
	($\lambda$x$_1$.$\lambda$y$_1$.$\lambda$z$_1$.y$_1$ ($\ldots$ 
		($\lambda$x$_{n-2}$.$\lambda$y$_{n-2}$.$\lambda$z$_n$.y$_{n-2}$ 
		true))) \\
	($\lambda$x$_1$.$\lambda$y$_1$.$\lambda$z$_1$.y$_1$ ($\ldots$ 
		$\lambda$y$_{n-2}$.$\lambda$z$_n$.y$_{n-2}$)) \\
	($\lambda$x$_1$.$\lambda$y$_1$.$\lambda$z$_1$.y$_1$ ($\ldots$ 
		($\lambda$x$_{n-3}$.$\lambda$y$_{n-3}$.$\lambda$z$_n$.y$_{n-3}$ 
		true))) \\
	Again note that if n = 2, we would be done here and have a result of 'true'. \\
	If n $\textgreater$ 3 this trend would continue recursively until resolving to true. \\

% Part C
\part

	As g resolves to false when n = 0, and true otherwise \\
	it then computes n $\textgreater$ 0 or n $\neq$ 0
	on the domain of natural numbers.

\problem{}

Consider the following \lambda$-expression$:

\begin{gather*}
Y : \lambda$t.($\lambda$x.(t (x x)) $\lambda$x.(t (x x)))$
\end{gather*}
Prove/disprove that (Y t) after application of several \beta$-reductions results in (t (Y t))$.

\solution

Setup \\
	\setlength\parindent{24pt}
	We are given (Y t), apply the definition of Y to this given lambda expression. \\
	(\lambda$t.($\lambda$x.(t (x x)) $\lambda$x.(t (x x)))$ $ t)$ \\
	Denote Q: \lambda $x.(t (x x))$ \\
	Note that (Q Q) after a single \beta$-reduction results in (t (Q Q))$ \\
	Thus we are starting with (t (Q Q)).

	\bigbreak
	\setlength\parindent{0pt}
	Claim \\
		\setlength\parindent{24pt}
		let n $\in$ $\lamtbb{N}$ \\
		If n $\beta$-reductions are taken of (t (Q Q)) denoted R \\
		then R $\neq$ (t (Y t)) \\
	 
	\bigbreak
	\setlength\parindent{0pt}
	Proof by Induction on n \\
	\setlength\parindent{24pt}
		Base Case (n = 0) \\
		\setlength\parindent{48pt}
		This is true as Q $\neq$ Y $\wedge$ Q $\neq$ t \\
		Thus (Q Q) $\neq$ (Y t) \\
		Therefore (t (Q Q)) $\neq $ (t (Y t)) \\
		\setlength\parindent{24pt}
		Induction Hypothesis \\
		\setlength\parindent{48pt}
		Assume ($t_0$ ($t_1$ ($\ldots$ ($t_n$ (Q Q))))) $\neq$ (t (Y t)) \\
		\setlength\parindent{24pt}
		Induction Step \\
		\setlength\parindent{48pt}
		We have ($t_0$ ($t_1$ ($\ldots$ ($t_n$ (Q Q))))) $\neq$ (t (Y t)) \\
		Apply a single $\beta$-reduction to the left hand side producing \\
		($t_0$ ($t_1$ ($\ldots$ ($t_n$ ($t_{n+1}$ (Q Q)))))) \\
		Note that ($t_0$ ($t_1$ ($\ldots$ ($t_n$ ($t_{n+1}$ (X)))))) $\neq$ (t (X)) for any X \\
		as ($t_n$ (\ldots)) is not a lambda expression that can be $\beta$-reduced. \\
		Therefore ($t_0$ ($t_1$ ($\ldots$ ($t_n$ ($t_{n+1}$ (Q Q)))))) $\neq$ (t (Y t)) \square\\

\end{document}
